\documentclass[../main.tex]{}
\begin{document}

\subsection{Clarke Wright Saving Algorithm}
The Clarke and Wright algorithm \cite{clarke_wright} is one of the earliest, and most popular heuristic algorithm for the VRP due to its speed, simplicity, and ease of adjustment to handle various constraints in real-life applications. Clarke Wright saving algorithm works equally well for both directed and undirected problems. The algorithm is built on a basic simple idea: maximize saving cost within a route, hence minimize total cost. Consider a depot D and n demand points. Suppose that initially the solution to the VRP consists of using n vehides and dispatching one vehicle to each one of the n demand points. The total cost of this solution is:
\[ C = 2\sum\limits_{i=1}^n d(D, i)\]
where d(i, j) is a function calculate distance between point i and , j, and C is total cost. To get a better solution, we now can combine 2 customers served by single vehicle on a single trip, the total distance we can save after combined is:
\begin{eqnarray}
s(i, j) & = & 2d(D, i) + 2d(D, j) \nonumber \\
&& -\: [d(D, i) + d(D, j) + d(i, j)] \nonumber \\
& = & d(D, i) + d(D, j) - d(i, j)
\end{eqnarray}•
\begin{algorithm}
\caption{Clarke Wright saving alorithm}\label{alg:clarke_wright}
\textbf{Step 1: Savings computation}
\flushleft
\begin{itemize}
\item Compute $s_{ij} \forall(i, j) \subset E$.
\item Sort $s_{ij}$
\end{itemize}
\textbf{Step 2:  Route Extension}
\flushleft
\begin{itemize}
\item for each route $(0, i, ..., j, 0)$
\item Determine the first saving ${s_{ki}}$ or ${s_{jl}}$ that can feasibly be used to merge the current route with another route ending with ${(k,0)}$ or starting with ${(0,l)}$.
\item Implement the merge and repeat this operation to the current route.
\item If not feasible merge exists, consider the next route and reapply the same operations.
\item Stop when not route merge is feasible.
\end{itemize}
\end{algorithm}

\subsection{Ruin and Recreate Algorithms}

Ruin and recreate framework is introduced in \cite{ref:rr_break}. By performing ruin and recreation on current solutions frequently enough, the better solutions will be obtained. A set of random vehicle routes are created and their total vehicle distance is calculated. The Ruin and Recreate approach is applied frequently to obtain another set of vehicle routes that have the least of total vehicle distance.

This approach starts with creating a set of initial vehicle routes: a random customer is selected, and combined with the depot location to create a base vector. The base vector starts to rotate clockwise with the depot location at the center and customers on the way are added to the route. If the route is over the vehicle capacity, a new route is created, and so on. Total vehicle distance is calculated for the initial routes.
Next, some customers from each initial route are removed by different ruin strategies and then added back to the same route in different order or different route using greedy insertion recreation strategy by trying all possible locations to pick the least total vehicle distance.
 There are three ruin strategies are used to remove customers from existing solution: random ruin, sequential ruin, and least customer ruin. Random strategy removes customers randomly from all customers of all routes, which creates the effect of diversifying the removing process. Sequential strategy removes customers sequentially in a selected route. The least customer strategy removes all customers in the route that has the least customers.
Greedy insertion strategy is used for the recreation of the solution by adding the removed customers back. This approach is to insert each removed customer to the cheapest possible route.

\begin{algorithm}
\caption{Ruin and recreate framework}\label{alg:RR}
\flushleft
\begin{itemize}
\item create intial solution
\item do {
\flushleft
\begin{itemize}
\item choose ruin and recreate strategy
\item ruin current solution
\item recreate ruined solution
\item if new solution better than current solution
\item then accept new solution
\item else accept new solution with probaility $e^{\frac{\Delta C}{T}}$
\end{itemize}
\item } while stop condition is not meet
\item return best solution
\end{itemize}
\end{algorithm}

\end{document}
